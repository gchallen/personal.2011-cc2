% <wc:start description="Content" max=800>

%I love Harvard. I have invested the last twelve years of my life in this
%complex place and don't regret a minute. Over time I have come to understand
%certain aspects of the College more clearly, and have discovered things that
%frustrate me. Find me in the dining hall at any given mealtime and I'm
%probably off about something: the way Harvard ignores teaching in the tenure
%process, the spiraling cost of a Harvard education and its link to growth in
%university bureaucracy, Harvard's reluctance to innovate or take risks
%despite its established position as the pinaccle of American higher
%education.

%Sometimes an exasperated interlocutor will ask me if there is anything I do
%like about Harvard. I love this place, and so the answer is yes. I'm proud
%that Harvard continues to improve its financial aid package. I'm proud that
%we took a public stand against ``Don't Ask, Don't Tell''. I'm proud Harvard
%admits and protects undocumented migrants. \sout{And I'm proud that Harvard
%has committed to providing opportunities for students from lower
%socioeconomic backgrounds by eliminating Early Action.}

Development Office, take note: your share of my future meager earnings has
taken a hit. Backpedaling on Dean of Admissions William R. Fitzsimmons ’67
claims that demolishing Early Action provided ``a win for students in the
bottom quarter and bottom half of the income distribution'' Harvard has
announced that it intends to scrap socioeconomic diversity in favor of
reentering the feeding frenzy over the ``most qualified'' students via early
admissions. Some of them might be---gasp---enrolling at Yale.

To examine the decision, let's take a trip down memory lane. Harvard
established its Early Action program in 1976. College counselors at selective
prep schools rejoiced. Marion Finbury, admissions counselor at Phillips
Andover Academy gushed that the program will be ``humane and marvelous'' for
the 50-odd students Phillips Andover sends to Harvard each year. But as early
as thirty-five years ago, Harvard admissions officers understand the impact
the move will have on the entering class. L. Fred Jewett '57, Dean of
Admissions, expects a to see far fewer minorities in the early pool.
Marvelous humaneness on offer to nervous students at a expensive prep school,
on hold for minorities.

%Fast forward to 1995 when  \textit{The Crimson} editorializes in favor of the
%program. Echoing 1994 comments by Director of Admissions Marlyn M. Lewis '70
%defending Early Action as a boon to Harvard's yield, \textit{The Crimson}
%also praises the program's yield and declares it necessary to maintain
%``magnificent classes of Harvardians for generations to come.'' Magnificent
%classes apparently do not include minorities or those from lower
%socioeconomic backgrounds.
%
%By 1998, however, \textit{The Crimson} is critical of the program, pointing
%out that ``early-action may be creating a two-tiered accepted group with
%different standards and different demographics.'' Harvard administrators
%declare that students demand Early Action, are ready to apply early, and
%eliminating the program would cause Harvard to ``lose by default many of the
%best students to other institutions that are able to capitalize on the
%national demand for early programs.''

Twenty-five years pass during which early admissions are periodically debated
along relatively static lines. In 2001, Richard Levin---then President of
that school in New Haven---initiates a national conversation on early
admissions programs. \textit{The Crimson} differentiates between binding and
non-binding early admissions programs, finding a way to praise Harvard and
attack Yale. However, in 2006 Harvard surprises everyone by announcing the
elimination of early admissions. Princeton and the University of Virginia
follow suit. Harvard professors studying admissions describe the move as ``a
selfless act, not some stratagem to outmaneuver its rivals.''

By 2008 \textit{The Crimson} declares the experiment a success, partly due to
2012 admissions being both highly-competitive (7.1\% admissions rate) and
producing the most diverse Harvard class in history. Harvard admissions
officers take to \textit{The Crimson} to laud the change, declaring that
``Harvard has begun a movement that will shape the conversation about access
and affordability.'' Come 2011 and the experiment began is over. Dean of the
Faculty Michael D. Smith declares that ``offering an accelerated decision
cycle for interested applicants will increase Harvard’s potential to attract
top-caliber students.''

For me, the story ends with confusion. Five years ago, admissions officers
proclaimed that eliminating Early Action would benefit applicants from lower
socioeconomic classes. Now that it is returning, they are claiming---wait for
it---that the move is intended to benefit competitive applicants from lower
socioeconomic classes: ``many highly talented students, including some of the
best-prepared low-income and underrepresented minority students, were
choosing programs with an early-action option''. Those two statements can't
be true simultaneously, and I don't think so much changed to allow them to be
true separated by five years. At the root of the confusion is a simple
question: what are we trying to accomplish by admitting students to the
College?

Harvard admissions is admittedly complex and no simple approach works in
isolation. What if Harvard admitted only the best students based on past
performance? Prepare yourself for a College full of students from wealthy
families who spent four years at private schools preparing their admissions
packets and taking SAT classes. Or maybe we should admit those students who
are going have a positive impact on the world? Define ``positive impact'',
develop a test to detect it and then get back to me. (Also notify the
Development Office that we won't be receiving any more donations from Wall
Street.) Should we attempt to field competitive sports teams and full
orchestras? Prepare for a sudden interest in squash and the bassoon among
college-savvy high schoolers. And what about a class that can broaden itself
through racial and socioeconomic diversity? Great idea! But wait---does it
conflict with our other goals?

At the end of the day, we have sixteen-hundred slots. How we parcel them out
says a lot about who we are and what we value: ``likely letters'' for
athletes, the ``Z-list'' for legacies. Many of our goals are conflicting and
we can't do everything, which means we have to have a frank conversation
about our priorities and the mission of this school. Without data and
transparency, I don't know whether Early Action was a boon to equalizing
access to Harvard or a bust. However, given the history of suspicion about
such programs, I have to conclude that Harvard has decided that competing
over top applicants is more important that broadening access to
underrepresented groups.

That's sad because I'm not worried about losing students to Yale or Princeton
or Dartmouth. Why should we? These are great schools where students receive
transformative educations. I am worried about losing the students that might
not have ever imagined Harvard---or Yale or Princeton or Stanford, or maybe
any college---could be for them, pay for them, and transform them. It seems
that zero-sum competition for over-prepared over-achievers has blinded us to
a larger context where universities have a collective role to play in
mobilizing education to combat rising economic inequality. For five years we
were brave. What are we now?

% <wc:end>

\textit{Geoffrey Challen '02--'03 is a Resident Tutor at Eliot House. The
views expressed are his and do not reflect official Harvard College policy.}
