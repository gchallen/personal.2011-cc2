% <wc:start description="Content" max=800>

%I love Harvard. I have invested the last twelve years of my life in this
%complex place and don't regret a minute. Over time I have come to understand
%certain aspects of the College more clearly, and have discovered things that
%frustrate me. Find me in the dining hall at any given mealtime and I'm
%probably off about something: the way Harvard ignores teaching in the tenure
%process, the spiraling cost of a Harvard education and its link to growth in
%university bureaucracy, Harvard's reluctance to innovate or take risks
%despite its established position as the pinaccle of American higher
%education.

%Sometimes an exasperated interlocutor will ask me if there is anything I do
%like about Harvard. I love this place, and so the answer is yes. I'm proud
%that Harvard continues to improve its financial aid package. I'm proud that
%we took a public stand against ``Don't Ask, Don't Tell''. I'm proud Harvard
%admits and protects undocumented migrants. \sout{And I'm proud that Harvard
%has committed to providing opportunities for students from lower
%socioeconomic backgrounds by eliminating Early Action.}

Now we know what keeps Harvard admissions officers up at night. It isn't the
threat of nuclear proliferation, or the fate of Palestine, or if the Sox will
have decent pitching this year, or even whether Brad will choose Chantal over
Emily. No. It's the fear that somewhere in America, a teenager may be making
a life-altering decision. Captured by an unfair system and swayed by
dishonest and one-sided information, they make a critical and permanent
error. They choose to attend Yale.

With the threat of this kind of calamity stalking the land, it's no surprise
that Harvard announced last week that Early Action will return next year.
Five years ago, Dean of Admissions William R. Fitzsimmons ’67 claimed that
cancelling Early Action provided ``a win for students in the bottom quarter
and bottom half of the income distribution''. Five years later, Harvard has
decided to rejoin the group of elite institutions participating in a zero-sum
competition over a small group of applicants rather than continue a
commitment to broadening and diversifying the overall applicant pool.

%Harvard has announced that it intends to scrap socioeconomic diversity in
%favor of reentering the feeding frenzy over the ``most qualified'' students
%via early admissions. Some of them might be---gasp---enrolling at Yale.

%To examine the decision, let's take a trip down memory lane. Harvard
%established its Early Action program in 1976. College counselors at selective
%prep schools rejoiced. Marion Finbury, admissions counselor at Phillips
%Andover Academy gushed that the program will be ``humane and marvelous'' for
%the 50-odd students Phillips Andover sends to Harvard each year. But as early
%as thirty-five years ago, Harvard admissions officers understood the impact
%the move would have on the entering class. L. Fred Jewett '57, Dean of
%Admissions, expects a to see far fewer minorities in the early pool.
%Marvelous humaneness on offer to nervous students at a expensive prep school,
%on hold for minorities.

%Fast forward to 1995 when \textit{The Crimson} editorializes in favor of the
%program. Echoing 1994 comments by Director of Admissions Marlyn M. Lewis '70
%defending Early Action as a boon to Harvard's yield, \textit{The Crimson}
%also praises the program's yield and declares it necessary to maintain
%``magnificent classes of Harvardians for generations to come.'' Magnificent
%classes apparently do not include minorities or those from lower
%socioeconomic backgrounds.
%
%By 1998, however, \textit{The Crimson} is critical of the program, pointing
%out that ``early-action may be creating a two-tiered accepted group with
%different standards and different demographics.'' Harvard administrators
%declare that students demand Early Action, are ready to apply early, and
%eliminating the program would cause Harvard to ``lose by default many of the
%best students to other institutions that are able to capitalize on the
%national demand for early programs.''

%Twenty-five years pass during which early admissions are periodically
%debated along relatively static lines. In 2001, Richard Levin---then
%President of that school in New Haven---initiates a national conversation on
%early admissions programs. \textit{The Crimson} differentiates between
%binding and non-binding early admissions programs, finding a way to praise
%Harvard and attack Yale. However, in 2006 Harvard surprises everyone by
%announcing the elimination of early admissions. Princeton and the University
%of Virginia follow suit. Harvard professors studying admissions describe the
%move as ``a selfless act, not some stratagem to outmaneuver its rivals.''
%
%By 2008 \textit{The Crimson} declares the experiment a success, partly due
%to 2012 admissions being both highly-competitive (7.1\% admissions rate) and
%producing the most diverse Harvard class in history. Harvard admissions
%officers take to \textit{The Crimson} to laud the change, declaring that
%``Harvard has begun a movement that will shape the conversation about access
%and affordability.'' Come 2011 and the experiment began is over. Dean of the
%Faculty Michael D. Smith declares that ``offering an accelerated decision
%cycle for interested applicants will increase Harvard’s potential to attract
%top-caliber students.''

Unfortunately, some of the most persuasive arguments against early admissions
have stubbornly refused to go away. A two-pool system where the pools look
different (diversity) and are treated differently (admissions rates) creates
the specter of unfairness, if not the reality. And a single application
deadline definitely gives admissions officers more time to tour high schools
in the fall, correcting common misconceptions about Harvard admissions and
touting our increasingly generous financial aid policy. None of Harvard's
statement surrounding the recent decision addressed these issues or the
central tenet underlying the 2006 decision: that early admissions provides an
unfair advantage to students from high-income families.

Unfortunately, that is probably still true: early admissions advantages the
advantaged. That said, these programs clearly provide a certain kind of
advantage to schools that employ them. When few other institutions followed
Harvard's lead and eliminated their early admissions programs, it forced us
to make a choice. Do we care more about re-arming ourselves in the
yield-driven competition with Yale and Princeton over the most-qualified
students, or are we willing to sacrifice some points on our yield in order to
bring new students into the application pool?

Now we know. And I, for one, am deeply disappointed. Step back and ask
yourself the question: do Harvard's yield and the competition over the best
of the best matter?

Consider the two pools of students this decision is balancing. The first pool
consists of well-prepared, highly-qualified, college-savvy applicants.
Frequently emerging from privileged backgrounds, they know the system and
will be able to leverage that knowledge and their own preparation to achieve
admission to an elite institution. Whether they end up at Harvard, Stanford,
or Yale is unknown, but that they will apply to Harvard, Stanford and Yale is
certain.

Given that this pool is large enough to supply Harvard many times over, the
fierce competition between institutions for these students is difficult to
justify. Despite Admissions Officer fears, my reaction to the nightmare
scenario in which a student chooses Yale over Harvard because of early
admissions is simple: so what? Yale is a great school! The student will
likely succeed there, and we will find a student willing to take a chance on
Harvard and fill their spot. How is this a bad outcome? The education we are
providing is not so different than what students receive at Yale or
Princeton, and if we were so demonstrably superior we would worry even less
about the quality of our entering student body. We would feel confident in
our institutional ability to transform even students from the bottom half of
the admissions pool into intelligent leaders and citizens.

The second pool consists of those that, despite being ready to be transformed
by an elite institution, lack the knowledge to navigate the college landscape
or our held back by misconceptions. Frequently emerging from less-privileges
backgrounds, they may apply to college but feel that their chances at
Harvard, Stanford or Yale or too low to merit applying. They, or their
parents, may worry that these schools are too expensive, or that they only
admit a certain kind of student. These students may never have a chance,
because they may never apply.

Unless you work in the Admissions Office or care about the inane U.S. News
college rankings, how well-qualified students sort out between excellent
institutions doesn't generate nightmares. Here's the future that keeps me up
at night. Too busy competing over the first pool of applicants, Harvard and
other peer institutions forget about the second group entirely. With poorer
students less-prepared and less likely to apply, admitted classes
increasingly perpetuate stereotypes and reinforce growing economic
inequality. With the educational economic elevator stuck at the top floor,
fifty years from now we may find that despite full financial aid the only
students capable of succeeding at Harvard come from the wealthiest 1\% of the
world's families, and that Harvard has reverted to what it was 100 years ago:
an expensive finishing school for the established elite.

Perhaps institutional competition for over-prepared over-achievers has
blinded us to a larger context where universities have a collective role to
play in mobilizing education to combat rising economic inequality. For five
years we were brave. What are we now?

% <wc:end>

\textit{Geoffrey Challen '02--'03 is a Resident Tutor at Eliot House. The
views expressed are his and do not reflect official Harvard College policy.}

%Given the quality of our applicant pool, Harvard could still field a class
%as accomplished and interesting even if our yield fell to 50\%, or even
%farther. Of course that class might not be the same. It might more closely
%mirror the racial and economic breakdown of the society in which we live,
%and that might actually be good. Our efforts to broaden the applicant pool
%might even benefit peer institutions while reenforcing the principle that
%access to higher education should be gated
%
%For me, the story ends with confusion. Five years ago, admissions officers
%proclaimed that eliminating Early Action would benefit applicants from lower
%socioeconomic classes. Now that it is returning, they are claiming---wait
%for it---that the move is intended to benefit competitive applicants from
%lower socioeconomic classes: ``many highly talented students, including some
%of the best-prepared low-income and underrepresented minority students, were
%choosing programs with an early-action option''. Those two statements can't
%be true simultaneously, and I don't think so much changed to allow them to
%be true separated by five years. At the root of the confusion is a simple
%question: what are we trying to accomplish by admitting students to the
%College?
%
%Harvard admissions is admittedly complex and no simple approach works in
%isolation. What if Harvard admitted only the best students based on past
%performance? Prepare yourself for a College full of students from wealthy
%families who spent four years at private schools preparing their admissions
%packets and taking SAT classes. Or maybe we should admit those students who
%are going have a positive impact on the world? Define ``positive impact'',
%develop a test to detect it and then get back to me. (Also notify the
%Development Office that we won't be receiving any more donations from Wall
%Street.) Should we attempt to field competitive sports teams and full
%orchestras? Prepare for a sudden interest in squash and the bassoon among
%college-savvy high schoolers. And what about a class that can broaden itself
%through racial and socioeconomic diversity? Great idea! But wait---does it
%conflict with our other goals?
%
%At the end of the day, we have sixteen-hundred slots. How we parcel them out
%says a lot about who we are and what we value: ``likely letters'' for
%athletes, the ``Z-list'' for legacies. Many of our goals are conflicting and
%we can't do everything, which means we have to have a frank conversation
%about our priorities and the mission of this school. Without data and
%transparency, it's hard to know whether Early Action was a boon to
%equalizing access to Harvard or a bust. But, given the history of suspicion
%about such programs, I have to conclude that Harvard has decided that
%competing over top applicants is more important that broadening access to
%underrepresented groups.
