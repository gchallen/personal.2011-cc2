% <wc:start description="Content" max=800>

%I love Harvard. I have invested the last twelve years of my life in this
%complex place and don't regret a minute. Over time I have come to understand
%certain aspects of the College more clearly, and have discovered things that
%frustrate me. Find me in the dining hall at any given mealtime and I'm
%probably off about something: the way Harvard ignores teaching in the tenure
%process, the spiraling cost of a Harvard education and its link to growth in
%university bureaucracy, Harvard's reluctance to innovate or take risks
%despite its established position as the pinaccle of American higher
%education.

Sometimes an exasperated interlocutor will ask me if there is anything I do
like about Harvard. I love this place, and the answer is yes. I'm proud that
Harvard continues to improve its financial aid package, leading the way for
other institutions to follow. I'm proud that we took a public stand against
``Don't Ask, Don't Tell'', a misguided and ultimately failed piece of
legislation. I'm proud Harvard admits and protects illegal immigrants,
refusing to punish talented children for their parents' mistakes. \sout{And
I'm proud that Harvard has committed to providing opportunities for students
from lower socioeconomic backgrounds by eliminating Early Action.}

Development Office, take note: your share of my future meager earnings has
taken a hit. Backpedaling on Dean of Admissions William R. Fitzsimmons ’67
claims that demolishing Early Action provided ``a win for students in the
bottom quarter and bottom half of the income distribution'' Harvard has
announced that it intends to scrap socioeconomic diversity in favor of
reentering the feeding frenzy over the ``most qualified'' students. Some of
them might be---gasp---enrolling at Yale.

To put the decision in context, let's take a trip down memory lane. In 1976
Harvard establishes its Early Action program. Marion Finbury, admissions
counselor at Phillips Andover Academy gushes that the program will be
``humane and marvelous'' for the 50-odd students Phillips Andover sends to
Harvard each year. L. Fred Jewett '57, Dean of Admissions, expects a to see
far fewer minorities in the newer pool. Marvelous humaneness on offer to
nervous students at a expensive prep school, on hold for minorities. Fast
forward to 1995 when  \textit{The Crimson} editorializes in favor of the
program. Echoing 1994 comments by Director of Admissions Marlyn M. Lewis '70
defending Early Action as a boon to Harvard's yield, \textit{The Crimson}
also praises the program's yield and declares it necessary to maintain
``magnificent classes of Harvardians for generations to come.'' Magnificent
classes apparently do not include minorities or those from lower
socioeconomic backgrounds.

By 1998, however, \textit{The Crimson} is critical of the program, pointing
out that ``early-action may be creating a two-tiered accepted group with
different standards and different demographics.'' Harvard administrators
declare that students demand Early Action, are ready to apply early, and
eliminating the program would cause Harvard to ``lose by default many of the
best students to other institutions that are able to capitalize on the
national demand for early programs.'' Three years later, Richard Levin---then
President of that school in New Haven---initiates a national conversation on
early admissions programs. \textit{The Crimson} differentiates between
binding and non-binding early admissions programs, cleverly finding a way to
praise Harvard while attacking Yale.

2006 dawns and Harvard surprises everyone by announcing the elimination of
early admissions. Princeton and the University of Virginia follow suit. Other
demur. Harvard praised for exercising moral leadership, with Harvard
professors describing the move as ``a selfless act, not some stratagem to
outmaneuver its rivals.'' By 2008 \textit{The Crimson} declares the
experiment a success, partly due to 2012 admissions being both
highly-competitive (7.1\% admissions rate) and producing the most diverse
Harvard class in history. Harvard admissions officers take to \textit{The
Crimson} to laud the change, declaring that ``Harvard has begun a movement
that will shape the conversation about access and affordability.''

Now, as quickly as the experiment began, it is over. Dean of the Faculty
Michael D. Smith declares that ``offering an accelerated decision cycle for
interested applicants will increase Harvard’s potential to attract
top-caliber students.'' Others make noises about improving diversity without
providing any evidence.

For me, this story ends with confusion. Five years ago, admissions officers
proclaimed that eliminating Early Action would benefit applicants from lower
socioeconomic classes. Now that it is returning, they are claiming---wait for
it---that this move will also benefit those from lower socioeconomic classes.
Those two statements can't be true simultaneously, and I don't think so much
changed even to allow them to be true separated by five years. Unfortunately,
no matter how you approach it, examining this question leads into the hot
mess of admissions decisions. Examining various approaches in isolation might
help expose some of the tensions underlying the overall process.

% <wc:end>

\textit{Geoffrey Challen '02--'03 is a Resident Tutor at Eliot House. The
views expressed are his and do not reflect official Harvard College policy.}
