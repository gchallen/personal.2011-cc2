% <wc:start description="Content" max=800>

Now we know what keeps Harvard admissions officers awake at night. It isn't
the threat of nuclear proliferation, the fate of Palestine, whether the Sox
can assemble a bullpen this year, or even whether Brad will choose Chantal
over Emily. No. It's the fear that somewhere in America, a teenager is making
a life-altering decision. Trapped in an unfair system and swayed by one-sided
information, they commit a critical and permanent error. They choose to
attend Yale.

With the threat of this kind of calamity stalking the land, it's no surprise
that Harvard recently announced the return of early admissions. Five years
ago, Dean of Admissions William R. Fitzsimmons `67 boasted in \textit{The
Crimson} that canceling Early Action provided ``a win for students in the
bottom quarter and bottom half of the income distribution''. And yet, five
years later, Harvard has decided to abandon the bottom half of the income
distribution to rejoin the group of elite institutions participating in a
zero-sum competition over the strongest members of the existing application
pool. Unfortunately, while Early Action is good for Harvard, it's bad for
America.

When Harvard ended early admissions five years ago the stated goal was the
opportunity to reach out to students that might not apply to Harvard or any
other selective institution. The single application deadline gave admissions
officers more time to tour high schools in the fall, correcting common
misconceptions, touting our generous financial aid packages, and encouraging
qualified students to apply. And eliminating a system with two pools that
looked different (socio-economically and racially) and were treated
differently (admissions rates) helped combat the specter of unfairness. When
you admit 20 percent of a whiter, richer group and 7 percent of everyone
e lse, it sends a message.

Now we are restarting early admissions, but none of the offered explanations
address the original goal. When Dean Fitzsimmons states---as he did recently
in \textit{The Crimson}---that ``we started to hear that more and more people
were applying early across the country'' and points to rising interest in
early admissions programs as a reason for the change, my response is: So
what? What does interest in early admissions programs have to do with the
diversity of the overall applicant pool? And when Dean of the Faculty of Arts
and Sciences Michael D. Smith says that ``many highly talented students,
including some of the best-prepared low-income and underrepresented minority
students, were choosing programs with an early-action option, and therefore
were missing out on the opportunity to consider Harvard'', my response is
again: So what? If anything, increasing numbers of low-income and
underrepresented students in the applicant pool might indicate our outreach
efforts are working!

In a way this evasiveness is not surprising, because while operating an Early
Action program drains resources from efforts at broadening our applicant
pool, it does achieve several things: It helps Harvard compete for students
already in the applicant pool, and boosts our all-important yield. But ignore
the warriors in the Harvard v. Yale admissions wars that nobody outside our
ivory towers cares about for a minute and ask the question: Do yield and the
competition over over-prepared students matter?

Not to me. Despite Admissions Office fears, my reaction to the nightmare
scenario in which a student chooses Yale over Harvard because of early
admissions is simple and familiar: So what? Yale is a great school. The
student will likely succeed there, and I'm guessing we will be able to find
\textit{someone} out of our thousands of qualified applicants willing to give
Harvard a shot. How is this a bad outcome? The education we are providing is
not so superior to what students at Yale or Princeton receive, and if it were
we would worry even less about the quality of admits, confident that by
graduation we would have transformed them into those peerless creatures:
Harvard Women and Men.

Unless you work in the Admissions Office or care about the inane U.S. News
college rankings, how well-qualified students sort out between Ivy League
schools doesn't generate nightmares or even a second thought. Here's the
nightmare scenario that keeps me awake. Consumed by competing over applicants
that would have applied anyway, Harvard and other peer institutions abandon
outreach efforts. Meanwhile, a qualified, lower-income student considers her
application. She's heard that Harvard is expensive. Her peers that have been
admitted are all wealthy. Discouraged, she pushes the application aside.
Harvard can never admit a student that doesn't apply.

% <wc:end>

\textit{Geoffrey Challen '02--'03 is a Resident Tutor at Eliot House. The
views expressed are his and do not reflect official Harvard College policy.}
