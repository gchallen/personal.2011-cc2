% <wc:start description="Content" max=800>

Now we know what keeps Harvard admissions officers up at night. It isn't the
threat of nuclear proliferation, the fate of Palestine, if the Sox will have
decent pitching this year, or even whether Brad will choose Chantal over
Emily. No. It's the fear that somewhere in America, a teenager is making a
life-altering decision. Captured by an unfair system and swayed by one-sided
information, they commit a critical and permanent error. They choose to
attend Yale.

With the threat of this kind of calamity stalking the land, it's no surprise
that Harvard recently announced the return of early admissions. Five years
ago, Dean of Admissions William R. Fitzsimmons ’67 boasted that canceling
Early Action provided ``a win for students in the bottom quarter and bottom
half of the income distribution''. Five years later, Harvard has decided to
rejoin the group of elite institutions participating in a zero-sum
competition over a small group of applicants and abandon our commitment to
broadening and diversifying the applicant pool.

Unfortunately, persuasive arguments against early admissions stubbornly
refuse to go away. A two-pool system where the pools look different
(socioeconomically and racially) and are treated differently (admissions
rates) creates the specter of unfairness, if not the reality. And a single
application deadline gives admissions officers more time to tour high schools
in the fall, correcting common misconceptions and touting our generous
financial aid policy. None of Harvard's statements surrounding the recent
decision addressed these issues or the central tenet underlying the 2006
decision: that early admissions provides an unfair advantage to students from
high-income families.

Unfortunately, that is probably still true: early admissions advantages one
group over another. The advantaged group consists of well-prepared,
highly-qualified, college-savvy applicants. Frequently emerging from
privileged backgrounds, they know the system and will be able to leverage
that knowledge and their own preparation to achieve admission to an elite
institution. Whether they end up at Harvard, Stanford, or Yale is unknown,
but that they will apply to Harvard, Stanford and Yale is certain.
Early admissions provides an advantage when pursuing these students, as
colleges can notify students earlier and begin to establish a relationship
before other schools release their decisions.

Given that this group is large enough to supply Harvard many times over, the
fierce competition between institutions for these students is difficult to
understand. Despite Admissions Officer fears, my reaction to the nightmare
scenario in which a student chooses Yale over Harvard because of early
admissions is simple: so what? Yale is a great school! The student will
likely succeed there, and we will find a student willing to take a chance on
Harvard and fill their spot. How is this a bad outcome? The education we are
providing is not so different than what students receive at Yale or
Princeton, and if it were so superior we would worry even less about the
quality of our entering student body.

The disadvantaged group consists of those that, despite being ready to be
transformed by an elite institution, lack the knowledge to navigate the
college landscape or are held back by misconceptions. Frequently emerging
from less-privileges backgrounds, they may apply to college but feel that
their chances at Harvard, Stanford or Yale are too low to merit applying.
They, or their parents, may worry that these schools are too expensive, or
that they only admit a certain kind of student. These students may never have
a chance, because they may never apply.

When few other institutions followed Harvard's lead and eliminated their
early admissions programs, it forced us to make a choice. Do we care more
about re-arming ourselves in the yield-driven competition with Yale and
Princeton over the most-qualified students, or are we willing to sacrifice
some points on our yield in order to bring new students into the application
pool? Now we know.

Step back from the Harvard v. Yale admissions wars that nobody outside our
ivory towers cares about for a minute and ask the question: why do yield and
the competition over over-prepared students matter? Unless you work in the
Admissions Office or care about the inane U.S. News college rankings, how
well-qualified students sort out between excellent institutions doesn't
generate nightmares. Here's the future that keeps me up at night. Too busy
competing over the first pool of applicants, Harvard and other peer
institutions neglect the second group. With poorer students less-prepared and
less likely to apply, admitted classes increasingly perpetuate stereotypes
and reinforce growing economic inequality. With the educational economic
elevator stuck at the top floor, fifty years from now we may find that
despite full financial aid the only students capable of succeeding at Harvard
come from the wealthiest 1\% of the world's families, and that Harvard has
reverted to what it was 100 years ago: an expensive finishing school for the
established elite.

Perhaps institutional competition for over-prepared over-achievers has
blinded us to a larger context where universities have a collective role to
play in mobilizing education to combat rising economic inequality. For five
years we were brave. What are we now?

% <wc:end>

\textit{Geoffrey Challen '02--'03 is a Resident Tutor at Eliot House. The
views expressed are his and do not reflect official Harvard College policy.}
